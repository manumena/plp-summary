% ALGUNOS PAQUETES REQUERIDOS (EN UBUNTU): %
% ========================================
% %
% texlive-latex-base %
% texlive-latex-recommended %
% texlive-fonts-recommended %
% texlive-latex-extra %
% texlive-lang-spanish (en ubuntu 13.10) %
% ******************************************************** %

\documentclass[a4paper]{article}
\usepackage[spanish]{babel}
\usepackage[utf8]{inputenc}
\usepackage{fancyhdr}
\usepackage[pdftex]{graphicx}
\usepackage{sidecap}
\usepackage{caption}
\usepackage{subcaption}
\usepackage{booktabs}
\usepackage{makeidx}
\usepackage{float}
\usepackage{amsmath, amsthm, amssymb}
\usepackage{amsfonts}
\usepackage{sectsty}
\usepackage{wrapfig}
\usepackage{listings}
\usepackage{enumitem}
\usepackage{hyperref}
\usepackage{listings}
\usepackage{listingsutf8}
\usepackage{enumitem}

% Para ver los marcos
% \usepackage{showframe}

\newcommand{\ord}{\ensuremath{\operatorname{O}}}
\newcommand{\nat}{\ensuremath{\mathbb{N}}}
\renewcommand{\thesubsubsection}{\thesubsection.\alph{subsubsection}}

% Lemas, definiciones, etc.
\theoremstyle{remark}
\newtheorem*{obs}{Observación}
\newtheorem*{nota}{Notación}
\newtheorem*{cor}{Corolario}
\theoremstyle{definition}
\newtheorem*{defi}{Definición}
\theoremstyle{plain}
\newtheorem{teo}{Teorema}
\newtheorem{lema}{Lema}
\newtheorem{prop}{Propiedad}
\usepackage{color} % para snipets de codigo coloreados
\usepackage{fancybox}  % para el sbox de los snipets de codigo

\definecolor{litegrey}{gray}{0.94}

% \newenvironment{sidebar}{%
% 	\begin{Sbox}\begin{minipage}{.85\textwidth}}%
% 	{\end{minipage}\end{Sbox}%
% 		\begin{center}\setlength{\fboxsep}{6pt}%
% 		\shadowbox{\TheSbox}\end{center}}
% \newenvironment{warning}{%
% 	\begin{Sbox}\begin{minipage}{.85\textwidth}\sffamily\lite\small\RaggedRight}%
% 	{\end{minipage}\end{Sbox}%
% 		\begin{center}\setlength{\fboxsep}{6pt}%
% 		\colorbox{litegrey}{\TheSbox}\end{center}}

\newenvironment{codesnippet}{%
	\begin{Sbox}\begin{minipage}{\textwidth}\sffamily\small}%
	{\end{minipage}\end{Sbox}%
		\begin{center}%
		\vspace{-0.4cm}\colorbox{litegrey}{\TheSbox}\end{center}\vspace{0.3cm}}



\usepackage{fancyhdr}
% \pagestyle{fancy}
%\renewcommand{\chaptermark}[1]{\markboth{#1}{}}
\renewcommand{\sectionmark}[1]{\markright{\thesection\ - #1}}
\fancyhf{}
% \fancyhead[LO]{Sección \rightmark} % \thesection\

% \fancyfoot[RO]{\thepage}
\renewcommand{\headrulewidth}{0.5pt}
\renewcommand{\footrulewidth}{0.5pt}
%\setlength{\hoffset}{-0.8in}
\setlength{\textwidth}{16cm}
\setlength{\hoffset}{-1.1cm}
\setlength{\headsep}{0.5cm}
\setlength{\textheight}{25cm}
\setlength{\voffset}{-0.7in}
\setlength{\headwidth}{\textwidth}
\setlength{\headheight}{13.1pt}
\renewcommand{\baselinestretch}{1.1} % line spacing


\begin{document}

\title{Paradigmas de Lenguajes de la Programación}
\author{Manuel Mena}
\maketitle

\tableofcontents

\newpage
\section{Practica 0}

\subsection{}
null :: Foldable t $\Rightarrow$ t a $\rightarrow$ Bool

Chequea si la estructura es nula \\

head :: [a] $\rightarrow$ a

Devuelve el primer elemento de la lista, mientras sea no vacía \\

tail :: [a] $\rightarrow$ a

Devuelve el último elemento de la lista, mientras sea no vacía \\

init :: [a] $\rightarrow$ [a]

Devuelve todos los elementos de la lista excepto el último, mientras sea no vacía \\

last :: [a] $\rightarrow$ [a]

Devuelve todos los elementos de la lista excepto el primero, mientras sea no vacía \\

take :: Int $\rightarrow$ [a] $\rightarrow$ [a]

Devuelve los primeros n elementos de la lista \\

drop :: Int $\rightarrow$ [a] $\rightarrow$ [a]

Devuelve una lista sin sus primeros n elementos \\

(++) :: [a] $\rightarrow$ [a] $\rightarrow$ [a]

Concatena ambas listas. Si la primera lista es infinita, devuelve la primera \\

concat :: Foldable t $\Rightarrow$ t [a] $\rightarrow$ [a]

The concatenation of all the elements of a container of lists \\

(!!) :: [a] $\rightarrow$ Int $\rightarrow$ a

Devuelve el n-ésimo elemento \\

elem :: Eq a $\Rightarrow$ a $\rightarrow$ t a $\rightarrow$ Bool

Devuelve si el elemento ocurre en la estructura

\end{document}
